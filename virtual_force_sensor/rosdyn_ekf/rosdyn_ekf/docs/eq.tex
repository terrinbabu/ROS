\documentclass{paper}
\onecolumn

\usepackage{amsmath} % assumes amsmath package installed
\usepackage{amssymb} % assumes amsmath package installed
\usepackage{graphicx}      % include this line if your document contains 
\DeclareMathOperator*{\sign}{sign}

\begin{document}

\title{rosdyn EKF}
\maketitle

\section{Dynamics}
Robot dynamics is normally model with an additive static friction (namely, the breakaway friction depends on $\sign(q)$.
However, static friction is not derivable when $\sign(q_i)=0$, which is required by the EKF.
To overcome this issue, a LuGre model is introduced. The LuGre model's internal state is not considered part of the observed state.

The motor torque is:
\begin{equation}
\tau=\tau_d+\tau_{f}(z,\dot{q})
\label{eq:tau}
\end{equation}
where $\tau_{f}(z,\dot{q})$ is the friction torque, $\tau_d$ is the component due to gravitation, Coriolis and inertial terms:
\begin{equation}
\tau_d=B(q)\ddot{q}+NL(q,\dot{q})
\label{eq:tau_dyn}
\end{equation}
where $B(q)$ is the inertia matrix, $NL(q,\dot{q})$ is the sum of gravitational part and Coriolis effect.

The forward dynamics equation is:
\begin{equation}
\ddot{q}=B(q)^{-1}\left(\tau_d-NL(q,\dot{q})-\tau_f(z,\dot{q})\right)
\end{equation}

The state equation is
\begin{equation}
\frac{d}{dt}\left[\begin{array}{c}
q\\
\dot{q}
\end{array}\right]
=
\left[\begin{array}{cc}
\dot{q}\\
B(q)^{-1}\left(\tau_d-NL(q,\dot{q})-\tau_f(z,\dot{q})\right)
\end{array}\right]
\end{equation}
\subsection{Jacobians}
Jacobians of (\ref{eq:tau_dyn}) w.r.t. $q$ and $\dot{q}$ are to complex to be computed analytically, therefore a numerical derivation is needed.
\begin{equation}
\frac{\partial\ddot{q}}{\partial q}=NUMERIAL
\end{equation}
\begin{equation}
\frac{\partial\ddot{q}}{\partial \dot{q}}=NUMERIAL
\end{equation}

Jacobian of (\ref{eq:tau_dyn}) w.r.t $\ddot{q}$ is the inertia matrix.
\begin{equation}
\frac{\partial\tau_d}{\partial \ddot{q}}=B(q)
\end{equation}



\section{LuGre Model}
Note: $z$ is supposed to be know. It is not part of observed states
\begin{equation}
\left\lbrace
\begin{aligned}
& g(\dot{q}) = c_0\\
& f(\dot{q}) = c_1 \dot{q}\\
\end{aligned}
\right.
\end{equation}
\begin{equation}
\left\lbrace
\begin{aligned}
& \dot{z} = \dot{q} - \sigma_0 \frac{\left|\dot{q}\right|}{c_0} z\\
& \tau_{f}(z,\dot{q}) = \sigma_0 z  + \sigma_1 \dot{z} + c_1\dot{q}.
\end{aligned}
\right.
\end{equation}

\begin{equation}
\left\lbrace
\begin{aligned}
& \dot{z} = \dot{q} - \sigma_0 \frac{\left|\dot{q}\right|}{c_0} z\\
& \tau_{f}(z,\dot{q}) = \sigma_0 z -\sigma_1\sigma_0 \frac{\left|\dot{q}\right|}{c_0} z + \sigma_1 \dot{q} + c_1\dot{q}.
\end{aligned}
\right.
\end{equation}


\begin{equation}
\left\lbrace
\begin{aligned}
& \dot{z} = \dot{q} - \sigma_0 \frac{\left|\dot{q}\right|}{c_0} z\\
& \tau_{f}(z,\dot{q}) = \sigma_0 z\left(I -\sigma_1 \frac{\left|\dot{q}\right|}{c_0}\right) + \sigma_1 \dot{q} + c_1\dot{q}.
\end{aligned}
\right.
\end{equation}



\subsection{Jacobians}

\begin{equation}
\begin{aligned}
\frac{\partial z}{\partial \dot{q}}&=I-\sign{\dot{q}}\frac{\sigma_0}{c_0}z\\
\frac{\partial z}{\partial z}&=-\sigma_0\frac{|\dot{q}|}{c_0}
\end{aligned}
\end{equation}


\begin{equation}
\begin{aligned}
\frac{\partial F}{\partial \dot{q}}&=-\frac{\sigma_0\sigma_1}{c_0}\sign(\dot{q}) z +\sigma_1+c_1\\
\frac{\partial F}{\partial z}&=\sigma_0\left(I -\sigma_1 \frac{\left|\dot{q}\right|}{c_0}\right)
\end{aligned}
\end{equation}

\subsection{Equilibrium point}

\begin{equation}
\dot{z} = \dot{q} - \sigma_0 \frac{\left|\dot{q}\right|}{c_0} z=0
\end{equation}

\begin{equation}
z=\frac{c_0 \sign(\dot{q})}{\sigma_0}
\end{equation}

\begin{equation}
\begin{aligned}
\tau_{f}(z,\dot{q}) &= c_0\sign(\dot{q}) \left(I -\sigma_1 \frac{\left|\dot{q}\right|}{c_0}\right) + \sigma_1 \dot{q} + c_1\dot{q}\\
 &= c_0\sign(\dot{q}) -\sigma_1 \dot{q} + \sigma_1 \dot{q} + c_1\dot{q}\\
 &= c_0\sign(\dot{q}) + c_1\dot{q}
\end{aligned}
\end{equation}



\end{document}
